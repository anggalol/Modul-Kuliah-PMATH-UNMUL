\chapter{Persamaan Kuadrat dan Fungsi Kuadrat}
\label{sec:second}

%% Subbab 1 %%

\section{Persamaan Kuadrat}

Persamaan adalah kalimat terbuka yang dihubungkan oleh tanda kesamaan ($ = $). Artinya, suatu persamaan belum diketahui nilai kebenarannya. Sebagai contoh, $ 2x + 3 = 5 $ merupakan persamaan. Kita tidak mengetahui apakah persamaan tersebut benar untuk beberapa nilai $ x $. Jika $ x = 1 $, maka persamaan tersebut bernilai benar sehingga persamaan tersebut menjadi suatu kesamaan (kalimat tertutup yang bernilai benar). Tetapi, jika $ x \ne 1 $, maka persamaan tersebut akan selalu bernilai salah sehingga persamaan tersebut menjadi suatu ketaksamaan (kalimat tertutup yang bernilai salah). Dalam hal ini, $ x = 1 $ kadang disebut sebagai solusi\footnote{Terkadang juga disebut sebagai akar atau penyelesaian.} dari persamaan $ 2x + 3 = 5 $.

\par Persamaan kuadrat merupakan salah satu dari banyak jenis persamaan dalam matematika. Persamaan kuadrat memiliki bentuk umum
\begin{equation} \label{eq:201}
	ax^{2} + bx + c = 0
\end{equation}
dengan $ a, b, c \in \mathbb{R} $ dan $ a \ne 0 $.

\begin{explbox}
	Mengapa dalam persamaan kuadrat haruslah $ a \ne 0 $?
\end{explbox}

\par Dalam persamaan kuadrat $ 3x^{2} + 2x - 5 = 0 $, nilai $ a = 3 $, $ b = 2 $, dan $ c = -5 $. Sedangkan pada persamaan kuadrat $ -3x^{2} + 10 = 0 $, nilai $ a = -3 $, nilai $ b = 0 $, dan nilai $ c = 10 $. Hal ini dikarenakan kita dapat menuliskan ekspresi $ -3x^{2} + 10 $ sebagai $ -3x^{2} + 0x + 10 $. Lalu bagaimana dengan persamaan $ x^{2} = -1 $?

\par Meskipun persamaan kuadrat memiliki bentuk umum seperti yang dapat dilihat pada persamaan \ref{eq:201}, persamaan seperti $ 2x + 3 = 0 $ juga dapat dipandang sebagai persamaan kuadrat, namun dalam $ \sqrt{x} $. Jika kita misalkan $ \sqrt{x} = u $, maka $ 2x + 3 = 0 $ dapat ditulis sebagai $ 2u^{2} + 3 = 0 $ yang merupakan persamaan kuadrat dalam $ u $ atau $ \sqrt{x} $.

\subsection{Cara Mencari Solusi Persamaan Kuadrat}

	Persamaan kuadrat tentunya memiliki solusi. Ingat kembali bahwa solusi adalah suatu nilai yang menyebabkan suatu persamaan menjadi kesamaan. Oleh karena itu, jika $ x = p $ merupakan solusi dari persamaan kuadrat umum \ref{eq:201}, maka $ ap^{2} + bp + c = 0 $. Tetapi, bagaimana cara menentukan solusinya? Dalam menyelesaikan suatu persamaan kuadrat, terdapat tiga cara yang dapat digunakan. Ketiga cara tersebut adalah dengan menggunakan faktorisasi, melengkapkan kuadrat, dan dengan menggunakan rumus kuadratik (rumus abc). Berikut merupakan penjelasannya.
	
	\subsubsection{Faktorisasi}
	
		Persamaan kuadrat umum seperti pada persamaan \ref{eq:201} dapat difaktorkan menjadi
		\begin{equation} \label{eq:202}
			a\left(x - p\right)\left(x - q\right) = 0.
		\end{equation}
		Disini, $ p $ dan $ q $ bisa jadi bukan termasuk bilangan real. Tetapi dalam buku ini, kita tidak akan membahas mengenai pemfaktoran dimana $ p $ dan $ q $ bukan bilangan real sehingga kita bisa anggap $ p, q \in \mathbb{R} $. Untuk lebih jauhnya mengenai hal ini akan dibahas pada bagian selanjutnya.
		
		\par Perhatikan kembali persamaan \ref{eq:202}. Jika kita substitusi $ x = p $, maka ruas kiri akan sama dengan nol. Tetapi, jika kita substitusi $ x = q $, maka ruas kiri juga akan menjadi nol. Oleh karena itu, $ p $ dan $ q $ merupakan akar-akar dari persamaan \ref{eq:202}. Dalam hal ini, kita tuliskan akar-akar dari persamaan \ref{eq:202} adalah $ x = p $ atau $ x = q $ (mengapa bukan 'dan'?). Jika kita pandang $ p $ sebagai akar pertama dan $ q $ sebagai akar kedua dari persamaan \ref{eq:202}, maka kita dapat tuliskan akar-akar dari persamaan $ \ref{eq:202} $ adalah $ x_{1} = p $ dan $ x_{2} = q $ (mengapa bukan 'atau'?).
		
		\par Pada persamaan \ref{eq:201}, jika $ a = 1 $, maka persamaan tersebut dapat ditulis sebagai
		\begin{equation} \label{eq:203}
			x^{2} + bx + c = 0.
		\end{equation}
		Perhatikan bahwa kita dapat memfaktorkan persamaan \ref{eq:203} menjadi
		\begin{equation} \label{eq:204}
			\left(x + p\right)\left(x + q\right) = 0
		\end{equation}
		dengan $ p $ dan $ q $ bilangan-bilangan real yang jika dijabarkan akan diperoleh
		\[ x^{2} + \left(p + q\right)x + pq = 0. \]
		Jadi, agar persamaan \ref{eq:203} ekuivalen dengan persamaan \ref{eq:204}, maka haruslah kita mencari $ p $ dan $ q $ sedemikian sehingga $ p + q = b $ dan $ pq = c $.
		
		\begin{contoh}
			Selesaikan persamaan kuadrat $ x^{2} - 7x + 10 = 0 $.
		\end{contoh}
		\begin{jawab}
			Jika kita ingin memfaktorkan ruas kiri persamaan pada soal menjadi $ \left(x + p\right)\left(x + q\right) = 0 $, maka kita harus mencari $ p $ dan $ q $ sedemikian sehingga $ p + q = -7 $ dan $ pq = 10 $. Setelah mencoba-coba, ternyata $ p = -2 $ dan $ q = -5 $ memenuhi. Oleh karena itu, kita dapat menulis ulang persamaan pada soal sebagai
			\[ \left(x - 2\right)\left(x - 5\right) = 0. \]
			Jadi, solusi persamaan kuadrat pada soal adalah $ x = 2 $ atau $ x = 5 $.
		\end{jawab}
		
		\par Pada persamaan \ref{eq:201}, jika $ a \ne 1 $, maka secara umum kita dapat menyelesaikannya dengan langkah-langkah sebagai berikut.
		\begin{enumerate}
			\item Pertama-tama, persamaan \ref{eq:201} kita faktorkan menjadi
			\begin{equation} \label{eq:205}
				\frac{\left(ax + p\right)\left(ax + q\right)}{a} = 0.
			\end{equation}
			\item Selanjutnya, carilah nilai $ p $ dan $ q $ sedemikian sehingga $ p + q = b $ dan $ pq = ac $.
			\item Setelah didapatkan nilai $ p $ dan $ q $, substitusikan kembali ke persamaan \ref{eq:205} dan sederhanakan lebih lanjut.
		\end{enumerate}
		
		\begin{explbox}
			Uji kebenaran langkah-langkah pemfaktoran di atas. Mengapa langkah-langkah tersebut benar?
		\end{explbox}
		
		\begin{contoh}
			Selesaikan persamaan kuadrat $ 2x^{2} - 5x - 3 = 0 $.
		\end{contoh}
		\begin{jawab}
			Pertama-tama, persamaan pada soal kita faktorkan menjadi
			\[ \frac{\left(2x + p\right)\left(2x + q\right)}{2} = 0. \]
			Selanjutnya, kita cari nilai $ p $ dan $ q $ sedemikian sehingga $ p + q = -5 $ dan $ pq = -6 $. Setelah mencoba-coba, ternyata $ p = 1 $ dan $ q = -6 $ memenuhi. Oleh karena itu, kita dapat menulis ulang pemfaktoran terakhir sebagai
			\[ \frac{\left(2x + 1\right)\left(2x - 6\right)}{2} = 0 \iff \left(2x + 1\right)\left(x - 3\right) = 0. \]
			Agar persamaan terakhir bernilai benar, maka $ 2x + 1 = 0 \iff x = -\dfrac{1}{2} $ atau $ x - 3 = 0 \iff x = 3 $.
			\par \noindent Jadi, solusi persamaan kuadrat pada soal adalah $ x = -\dfrac{1}{2} $ atau $ x = 3 $.
		\end{jawab}
		
		\begin{warningbox}
			Jika Anda ingin mencari solusi dari persamaan $ x^{2} = 9 $, Anda tidak boleh langsung menarik akar pada kedua ruas sehingga didapatkan $ x = 3 $. Hal ini dikarenakan, $ x = -3 $ juga memenuhi persamaan tersebut. Beberapa cara untuk memunculkan solusi $ x = -3 $ adalah dengan menambahkan tanda '$ \pm $' di ruas kanan persamaan setelah ditarik akarnya seperti $ x^{2} = 9 \implies x = \pm 3 $ sehingga solusinya adalah $ x = 3 $ atau $ x = -3 $. Anda juga dapat mengurangi kedua ruas dengan tiga, lalu memfaktorkannya dengan identitas selisih kuadrat seperti
			\[ x^{2} = 9 \iff x^{2} - 9 = 0 \iff \left(x + 3\right)\left(x - 3\right) = 0 \]
			sehingga solusinya adalah $ x = 3 $ atau $ x = -3 $.
		\end{warningbox}
		
	\subsubsection{Melengkapkan Kuadrat}
	
		Terkadang, kita akan kesulitan untuk memfaktorkan suatu bentuk kuadrat. Sebagai contoh, Anda dapat mencoba memfaktorkan ekspresi $ x^{2} + x + 1 $ ini. Anda pasti akan kesulitan memfaktorkan ekspresi tersebut. Oleh karena itu, teknik melengkapkan kuadrat akan sangat membantu disini. Teknik ini sangatlah berguna untuk menyelesaikan suatu persamaan kuadrat yang sukar untuk difaktorkan.
		
		\par Melengkapkan kuadrat adalah suatu teknik untuk mengubah bentuk persamaan kuadrat umum seperti pada persamaan \ref{eq:201} menjadi suatu persamaan kuadrat berbentuk
		\begin{equation} \label{eq:206}
			a\left(x + h\right)^{2} + k = 0.
		\end{equation}
		untuk suatu bilangan real $ h $ dan $ k $. Dalam bentuk ini, persamaan kuadrat akan jauh lebih mudah untuk diselesaikan dibandingkan pada bentuk umumnya.
		
		\par Jika $ a = 1 $, tentunya proses melengkapkan kuadrat akan mudah. Misalkan $ a = 1 $, maka persamaan \ref{eq:201} dapat dituliskan sebagai
		\[ x^{2} + bx + c = 0 \]
		seperti yang terdapat pada persamaan \ref{eq:203}.
		\par \noindent Perhatikan bahwa $ \left(x + \dfrac{1}{2}b\right)^{2} = x^{2} + bx + \left(\dfrac{b}{2}\right)^{2} $. Oleh karena itu, kita harus munculkan bentuk $ \left(\dfrac{b}{2}\right)^{2} $ pada persamaan \ref{eq:203} dengan menjumlahkan kedua ruas dengan bentuk tersebut. Oleh karena itu, persamaan \ref{eq:203} dapat kita tulis ulang sebagai
		\[ x^{2} + bx + \left(\frac{b}{2}\right)^{2} + c = \left(\frac{b}{2}\right)^{2} \iff \left(x + \frac{1}{2}b\right)^{2} + c - \left(\frac{b}{2}\right)^{2} = 0. \]
		Dengan membandingan koefisien-koefisien ruas kiri persamaan terakhir dengan ruas kiri persamaan \ref{eq:206}, kita dapatkan $ h = \dfrac{b}{2} $ dan $ k = c - \left(\dfrac{b}{2}\right)^{2} $.
		
		\begin{contoh}
			Dengan melengkapkan kuadrat, tentukan penyelesaian dari persamaan $ x^{2} + 4x - 9 = 0 $.
		\end{contoh}
		\begin{jawab}
			Karena $ b = 4 $ dan $ c = -9 $, maka $ h = \dfrac{1}{2}\left(4\right) = 2 $. Oleh karena itu, kita bisa menjumlahkan kedua ruas dengan $ 2^{2} = 4 $ sehingga
			\[ x^{2} + 4x + 4 - 9 = 4 \iff \left(x + 2\right)^{2} = 13. \]
			Menarik akar kedua ruas akan didapatkan
			\[ x + 2 = \pm 13 \iff x = -2 \pm 13. \]
			Jadi solusi persamaan kuadrat pada soal adalah $ x = -2 - \sqrt{13} $ atau $ x = -2 + \sqrt{13} $.
		\end{jawab}
		
		\par Jika $ a \ne 1 $, kita dapat mencoba untuk membagi kedua ruas persamaan kuadrat dengan $ a $. Sebagai contoh, kita akan mengerjakan soal berikut.
		
		\begin{contoh}
			Dengan melengkapkan kuadrat, tentukan penyelesaian dari persamaan $ 2x^{2} + 3x + 1 = 0 $.
		\end{contoh}
		\begin{jawab}
			Dengan membagi kedua ruas dengan 2, kita akan mendapatkan
			\[ x^{2} + \frac{3}{2}x + \frac{1}{2} = 0. \]
			Sampai disini, proses untuk melengkapkan kuadrat sama seperti contoh sebelumnya dan diserahkan kepada pembaca sebagai latihan.
		\end{jawab}
		
		\par Terkadang, membagi kedua ruas dengan $ a $ akan menimbulkan masalah baru seperti rumitnya proses melengkapkan kuadrat karena harus berurusan dengan pecahan yang bentuknya ekstrim. Oleh karena itu, alih-alih membagi kedua ruas dengan $ a $, kita juga bisa mencoba untuk mengalikan kedua ruas dengan $ a $ sehingga akan didapatkan persamaan kuadrat baru dalam $ ax $. Sebagai contoh, kita akan mengerjakan soal berikut.
		
		\begin{contoh}
			Dengan melengkapkan kuadrat, tentukan penyelesaian dari persamaan $ 3x^{2} + x - 5 = 0 $.
		\end{contoh}
		\begin{jawab}
			Dengan mengalikan kedua ruas dengan 3, kita akan mendapatkan
			\[ 3^{2}x^{2} + 3x - 5 = 0 \iff \left(3x\right)^{2} + 3x - 5 = 0. \]
			Misalkan $ 3x = u $, maka kita bisa tulis ulang persamaan terakhir sebagai
			\[ u^{2} + u - 5 = 0. \]
			Sampai disini, proses untuk melengkapkan kuadrat sama seperti contoh sebelumnya dan diserahkan kepada pembaca sebagai latihan. Jangan lupa untuk substitusi balik $ u = 3x $.
		\end{jawab}
	
	\subsubsection{Rumus Kuadratik}
		
		Di Indonesia, rumus kuadratik sering disebut sebagai rumus abc. Rumus kuadratik sebenarnya didapatkan dengan menggunakan teknik melengkapkan kuadrat pada persamaan \ref{eq:201}. Rumus kuadratik dapat dituliskan sebagai berikut.
		
		\begin{teorema}[Rumus Kuadratik] \label{thm:21}
			Misalkan solusi-solusi dari persamaan \ref{eq:201} adalah $ x_{1, 2} $ (yang berarti $ x_{1} $ dan $ x_{2} $), maka
			\begin{equation} \label{eq:207}
				x_{1, 2} = \frac{-b \pm \sqrt{b^{2} - 4ac}}{2a}.
			\end{equation}
		\end{teorema}
		\begin{bukti}
			Membagi kedua ruas persamaan \ref{eq:201} dengan $ a $ dan dilanjutkan dengan melengkapkan kuadrat akan didapatkan
			\begin{alignat*}{2}
				&\qquad& x^{2} + \frac{b}{a}x + \frac{c}{a} &= 0 \\
				\iff&& x^{2} + \frac{b}{a}x + \left(\frac{b}{2a}\right)^{2} + \frac{c}{a} &= \left(\frac{b}{2a}\right)^{2} \\
				\iff&& \left(x + \frac{b}{2a}\right)^{2} &= \frac{b^{2}}{4a^{2}} - \frac{c}{a} \\
				\iff&& \left(x + \frac{b}{2a}\right)^{2} &= \frac{b^{2}a - 4a^{2}c}{4a^{3}} \\
				\iff&& \left(x + \frac{b}{2a}\right)^{2} &= \frac{b^{2} - 4ac}{4a^{2}} \\
				\iff&& x_{1, 2} + \frac{b}{2a} &= \pm \frac{\sqrt{b^{2} - 4ac}}{2a} \\
				\iff&& x_{1, 2} &= \frac{-b \pm \sqrt{b^{2} - 4ac}}{2a}
			\end{alignat*}
			dan kita selesai.
		\end{bukti}
	
		\begin{explbox}
			Alih-alih membagi kedua ruas dengan $ a $, coba Anda buktikan teorema \ref{thm:21} dengan mengalikan kedua ruas persamaan \ref{eq:201} dengan $ a $, lalu dilanjutkan dengan melengkapkan kuadrat.
		\end{explbox}
		
		\par Rumus kuadratik ini merupakan alat yang sangat canggih untuk menyelesaikan suatu persamaan kuadrat. Mungkin kita akan kesulitan jika menyelesaikan persamaan kuadrat dengan menggunakan pemfaktoran atau melengkapkan kuadrat, tetapi rumus kuadratik ini akan dapat membantu kita dengan cepat. Rumus kuadratik inilah yang akan membantu kita untuk memahami persamaan kuadrat lebih lanjut lagi.
		
		\begin{contoh}
			Dengan menggunakan rumus kuadratik, tentukan akar-akar dari persamaan $ 2x^{2} - x - 11 = 0 $.
		\end{contoh}
		\begin{jawab}
			Pertama, kita tentukan terlebih dahulu nilai $ a $, $ b $, dan $ c $. Dengan membandingkan koefisien-koefisien persamaan pada soal dengan persamaan \ref{eq:201}, kita dapatkan $ a = 2 $, $ b = -1 $, dan $ c = -11 $. Oleh karena itu, dengan rumus kuadratik, kita akan mendapatkan
			\[ x_{1, 2} = \frac{-\left(-1\right) \pm \sqrt{\left(-1\right)^{2} - 4\left(2\right)\left(-11\right)}}{2\left(2\right)} = \frac{1 \pm \sqrt{1 + 88}}{4} = \frac{1 \pm \sqrt{89}}{4} \]
			sehingga akar-akar persamaan pada soal adalah
			\[ x_{1} = \frac{1 + \sqrt{89}}{4} \quad \mbox{dan} \quad x_{2} = \frac{1 - \sqrt{89}}{4}. \]
		\end{jawab}
		
		\par Dari rumus kuadratik ini, kita dapat memfaktoran bentuk kuadrat $ ax^{2} + bx + c $ sebagai
		\[ a\left(x - \frac{-b + \sqrt{b^{2} - 4ac}}{2a}\right)\left(x - \frac{-b - \sqrt{b^{2} - 4ac}}{2a}\right) \]
		dan pada kasus khusus jika $ b^{2} = 4ac $, kita dapat memfaktorkan bentuk kuadrat $ ax^{2} + bx + c $ sebagai
		\[ a\left(x + \frac{b}{2a}\right)^{2}. \]
		
		\begin{explbox}
			Coba cari tahu mengenai persamaan kuadratik tereduksi. Bagaimanakah rumus kuadratik digunakan dalam persamaan tersebut?
		\end{explbox}
	
\subsection{Diskriminan}

	Dalam persamaan kuadrat, diskriminan adalah ekspresi yang berada di bawah tanda akar pada rumus kuadratik seperti pada teorema \ref{thm:21}. Diskriminan sering disimbolkan sebagai huruf kapital $ D $ atau huruf kapital Yunani $ \Delta $ (delta kapital)\footnote{Di Indonesia sendiri, notasi yang sering digunakan untuk diskriminan adalah $ D $ sehingga selanjutnya kita akan menotasikan diskriminan sebagai $ D $, kecuali ada keterangan lain.} dimana
	\begin{equation} \label{eq:208}
		D = \Delta = b^{2} - 4ac.
	\end{equation}
	dengan $ a, b, c $ berturut-turut merupakan koefisien-koefisien persamaan kuadrat umum \ref{eq:201}. Oleh karena itu, formula \ref{eq:207} dapat dituliskan sebagai
	\begin{equation} \label{eq:209}
		x_{1, 2} = \frac{-b \pm \sqrt{D}}{2a}.
	\end{equation}
	
\subsection{Jenis-jenis Akar Persamaan Kuadrat}

	Terdapat dua kemungkinan akar-akar yang mungkin untuk persamaan kuadrat. Dua kemungkinan ini bergantung pada nilai diskriminan pada persamaan kuadrat tersebut. Jika $ D < 0 $, maka akar-akar persamaan kuadrat tersebut bukan merupakan bilangan real (bilangan kompleks, yang kadang dinotasikan sebagai $ \mathbb{C} $), tetapi jika $ D \geq 0 $, maka akar-akar persamaan kuadrat tersebut merupakan bilangan real.
	
	\par Jika $ D < 0 $, maka nilai di dalam akar pada persamaan \ref{eq:209} akan bernilai negatif. Padahal nilai di dalam akar harus nonnegatif agar nilainya terdefinisi pada garis bilangan real. Oleh karena itu, akar-akar yang dihasilkan oleh persamaan kuadrat yang diskriminannya negatif akan bernilai tidak real. Lain halnya jika $ D \geq 0 $. Nilai di dalam akar pada persamaan \ref{eq:209} akan bernilai nonnegatif yang sudah pasti nilainya akan terdefinisi pada garis bilangan real.
	
	\par Biasanya jika suatu persamaan kuadrat memiliki diskriminan negatif, kita tidak akan mencari akar-akarnya secara eksplisit. Hal ini dikarenakan ruang lingkup pembicaraan kita yang terbatas hanya untuk bilangan real. Tetapi jika Anda memiliki keingintahuan yang tinggi, Anda dapat mencoba untuk menyelesaikannya dengan menggunakan rumus kuadratik. Tetapi untuk menyelesaikannya, Anda perlu mengetahui terlebih dahulu mengenai bilangan imajiner. Sebagai contoh, misalkan kita diberikan suatu persamaan kuadrat $ x^{2} + x + 1 = 0 $. Diskriminan persamaan kuadrat tersebut adalah $ D = 1^{2} - 4\left(1\right)\left(1\right) = -3 $. Oleh karena itu, dengan menggunakan rumus kuadratik kita akan mendapatkan
	\[ x_{1, 2} = \frac{-1 \pm \sqrt{-3}}{2}. \]
	Untuk menghadapi akar dengan ekspresi di dalam akar yang bernilai negatif, kita perlu untuk menguraikannya terlebih dahulu. Dalam hal ini, kita uraikan $ -3 $ sebagai $ -1 \cdot 3 $. Oleh karena itu, $ \sqrt{-3} = \sqrt{-1 \cdot 3} = \sqrt{3} \cdot \sqrt{-1} $. Dalam matematika, biasanya kita notasikan $ \sqrt{-1} $ sebagai $ i $ sehingga $ \sqrt{3} \cdot \sqrt{-1} = \sqrt{3}i $. Bilangan $ i $ inilah yang disebut sebagai bilangan imajiner. Oleh karena itu,
	\[ x_{1, 2} = \frac{-1 \pm \sqrt{3}i}{2} \]
	sehingga akar-akar persamaan kuadrat $ x^{2} + x + 1 = 0 $ adalah
	\[ x_{1} = \frac{-1 + \sqrt{3}i}{2} \quad \mbox{dan} \quad x_{2} = \frac{-1 - \sqrt{3}i}{2}. \]
	
	\par Persamaan kuadrat yang memiliki diskriminan nonnegatif dibagi lagi menjadi dua jenis, yaitu persamaan kuadrat dengan diskriminan nol dan persamaan kuadrat dengan diskriminan positif. Jika suatu persamaan kuadrat memiliki nilai $ D = 0 $, maka persamaan \ref{eq:209} dapat dituliskan sebagai
	\[ x_{1, 2} = -\frac{b \pm 0}{2a} = -\frac{b}{2a}. \]
	Akibatnya, $ x_{1} = x_{2} = -\dfrac{b}{2a} $ sehingga persamaan kuadrat tersebut memiliki tepat satu akar real. Terkadang juga disebut akarnya berulang atau akarnya ganda. Hal ini akan dibahas pada bagian tersendiri untuk lebih memahaminya lebih lanjut. Selain itu, jika suatu persamaan kuadrat memiliki nilai $ D > 0 $, maka persamaan kuadrat tersebut memiliki dua akar yang berbeda, yaitu
	\[ x_{1} = -\frac{-b + \sqrt{D}}{2a} \quad \mbox{dan} \quad x_{2} = \frac{-b - \sqrt{D}}{2a} \]
	atau kebalikannya.
	
	\begin{explbox}
		\begin{enumerate}[nosep, leftmargin=*]
			\item Apakah mungkin salah satu akar dari suatu persamaan kuadrat merupakan bilangan real, sedangkan akar yang lainnya merupakan bilangan takreal?
			\item Apakah mungkin salah satu akar dari suatu persamaan kuadrat merupakan bilangan rasional, sedangkan akar yang lainnya merupakan bilangan irasional?
		\end{enumerate}
	\end{explbox}
	
\subsection{Bentuk-bentuk Simetris Akar-Akar Persamaan Kuadrat}
	
	Terdapat suatu hubungan antara $ x_{1} $ dengan $ x_{2} $. Hubungan tersebut merupakan akibat langsung dari rumus kuadratik. Dari formula \ref{eq:207}, misalkan
	\[ x_{1} = \frac{-b + \sqrt{b^{2} - 4ac}}{2a} \quad \mbox{dan} \quad x_{2} = \frac{-b - \sqrt{b^{2} - 4ac}}{2a}. \]
	Jika kita jumlahkan $ x_{1} $ dengan $ x_{2} $, kita akan mendapatkan
	\[ x_{1} + x_{2} = \frac{-b + \ccancel[blue]{\sqrt{b^{2} - 4ac}} + \left(-b - \ccancel[blue]{\sqrt{b^{2} - 4ac}}\right)}{2a} = \frac{-2b}{2a} \]
	sehingga
	\begin{equation} \label{eq:210}
		x_{1} + x_{2} = -\frac{b}{a}.
	\end{equation}
	Selain itu, jika kita kalikan $ x_{1} $ dengan $ x_{2} $, kita akan mendapatkan
	\[ x_{1}x_{2} = \frac{\left(-b + \sqrt{b^{2} - 4ac}\right)\left(-b - \sqrt{b^{2} - 4ac}\right)}{2a \cdot 2a} = \frac{\ccancel[blue]{b^{2}} - \left(\ccancel[blue]{b^{2}} - 4ac\right)}{4a^{2}} = \frac{4ac}{2a} \]
	sehingga
	\begin{equation} \label{eq:211}
		x_{1}x_{2} = \frac{c}{a}.
	\end{equation}
	Hal ini tidak menjadi masalah apabila kita misalkan $ x_{1} $ dan $ x_{2} $ sebagai kebalikannya. Nilai dari $ x_{1} + x_{2} $ dan $ x_{1}x_{2} $ akan tetap sama meskipun $ x_{1} $ dan $ x_{2} $ dipertukarkan pemisalannya.
	
	\begin{infobox}{Informasi}
		Jumlah dan hasil kali akar-akar persamaan kuadrat ini merupakan kasus khusus dari teorema Vieta untuk polinomial berderajat dua (bentuk kuadrat). Anda dapat membacanya lebih lanjut di \\
		https://en.wikipedia.org/wiki/Vieta\%27s\_formulas.
	\end{infobox}
	
	\par Pertukaran pemisalan $ x_{1} $ dan $ x_{2} $ akan menjadi masalah apabila kita ingin menghitung nilai dari $ x_{1} - x_{2} $. Jika kita kurangi $ x_{1} $ dengan $ x_{2} $ kita akan mendapatkan
	\[ x_{1} - x_{2} = \frac{\ccancel[blue]{-b} + \sqrt{b^{2} - 4ac} - \left(\ccancel[blue]{-b} - \sqrt{b^{2} - 4ac}\right)}{2a} = \frac{2\sqrt{b^{2} - 4ac}}{2a}. \]
	Karena $ D = b^{2} - 4ac $, maka kita punyai
	\begin{equation} \label{eq:212}
		x_{1} - x_{2} = \frac{\sqrt{D}}{a}.
	\end{equation}
	Tetapi, jika kita misalkan kebalikannya ($ x_{1} $ dan $ x_{2} $ dipertukarkan), yaitu
	\[ x_{1} = \frac{-b - \sqrt{b^{2} - 4ac}}{2a} \quad \mbox{dan} \quad x_{2} = \frac{-b + \sqrt{b^{2} - 4ac}}{2a}, \]
	maka
	\begin{equation} \label{eq:213}
		x_{1} - x_{2} = -\frac{\sqrt{D}}{a}.
	\end{equation}
	Oleh karena itu, secara umum, jika kita tidak mengetahui mana yang $ x_{1} $ dan mana yang $ x_{2} $, maka kemungkinan nilai-nilai dari $ x_{1} - x_{2} $ adalah
	\begin{equation} \label{eq:214}
		x_{1} - x_{2} = \pm \frac{\sqrt{D}}{a}.
	\end{equation}
	Hal ini berlaku untuk sebarang $ x_{1} $ dan $ x_{2} $, bahkan ketika keduanya bukan merupakan bilangan real ($ D < 0 $). Ingat bahwa meskipun ada tanda '$ \pm $' di ruas kanan, ini bukan berarti terdapat dua solusi, tetapi yang lebih tepat adalah terdapat dua kemungkinan nilai $ x_{1} - x_{2} $. Anda perlu mengecek kembali kondisi-kondisi yang diberikan pada soal karena bisa saja kemungkinan yang lain tidak memenuhi kondisi yang diberikan pada soal. Anda hanya boleh menuliskan tanda '$ \pm $' jika tidak ada kondisi lebih lanjut yang diberikan pada soal.
	
	\par Untuk kasus khusus ketika $ x_{1} $ dan $ x_{2} $ real, yaitu ketika $ D \geq 0 $, maka $ \sqrt{D} $ akan bernilai real sehingga persamaan \ref{eq:214} ekuivalen dengan\footnote{Penjelasan mengenai hal ini akan dipelajari pada bab 3.}
	\begin{equation} \label{eq:215}
		\left|x_{1} - x_{2}\right| = \left|\frac{\sqrt{D}}{a}\right| = \frac{\sqrt{D}}{\left|a\right|}.
	\end{equation}
	Hal ini tidak berlaku untuk $ D < 0 $. Alasannya adalah, jika $ D < 0 $, maka $ \sqrt{D} $ bukan merupakan bilangan real. Permasalahannya adalah, nilai mutlak dari bilangan takreal tidak didefinisikan pada buku ini. Meskipun ada definisinya pada buku matematika lanjut, kita tidak akan membahas mengenai hal tersebut.
	
	\begin{explbox}
		Bagaimana dengan $ \dfrac{x_{1}}{x_{2}} $?
	\end{explbox}
	
	\begin{contoh}
		Diberikan persamaan kuadrat $ 2x^{2} - x + 5 = 0 $. Tentukan nilai dari
		\begin{enumerate}
			\item $ x_{1} + x_{2} $
			\item $ x_{1}x_{2} $
			\item $ \left|x_{1} - x_{2}\right| $
			\item $ x_{1}^{2} - x_{2}^{2} $
		\end{enumerate}
	\end{contoh}
	\begin{jawab}
		Pertama, kita tentukan terlebih dahulu nilai $ a $, $ b $, dan $ c $. Dengan membandingkan koefisien-koefisien persamaan pada soal dengan koefisien-koefisien persamaan kuadrat umum \ref{eq:201}, kita akan mendapatkan $ a = 2 $, $ b = -1 $, dan $ c = 5 $. Oleh karena itu,
		\begin{enumerate}
			\item Dari formula \ref{eq:210}, $ x_{1} + x_{2} = -\dfrac{b}{a} = -\dfrac{-1}{2} = \dfrac{1}{2} $.
			\item Dari formula \ref{eq:211}, $ x_{1}x_{2} = \dfrac{c}{a} = \dfrac{5}{2} $.
			\item Perhatikan bahwa $ D = b^{2} - 4ac = \left(-1\right)^{2} - 4\left(2\right)\left(5\right) = 1 - 40 = -39 $ sehingga berdasarkan formula \ref{eq:215}, nilai dari $ \left|x_{1} - x_{2}\right| $ tidak didefinisikan.
			\item Dengan menggunakan formula \ref{eq:214}, kita punyai $ x_{1} - x_{2} = \pm \dfrac{\sqrt{D}}{a} = \pm \dfrac{\sqrt{39}i}{2} $ sehingga
			\[ x_{1}^{2} - x_{2}^{2} = \left(x_{1} + x_{2}\right)\left(x_{1} - x_{2}\right) = \frac{1}{2} \cdot \pm \frac{\sqrt{39}i}{2} = \pm \frac{\sqrt{39}i}{4}. \]
		\end{enumerate}
	\end{jawab}
	
	\par Dari soal di atas, kita bisa mengamati bahwa meskipun diskriminan suatu persamaan kuadrat bernilai negatif, nilai dari $ x_{1} + x_{2} $ dan $ x_{1}x_{2} $ tetap bernilai real. Hal ini dikarenakan $ x_{1} + x_{2} $ dan $ x_{1}x_{2} $ tidak bergantung pada nilai diskriminan, tetapi hanya koefisien-koefisien persamaan kuadrat tersebut. Lain halnya dengan $ x_{1} - x_{2} $, ekspresi tersebut bergantung kepada nilai diskriminan. Apabila diskriminannya negatif, $ x_{1} - x_{2} $ akan bernilai takreal. Nilai dari $ \left|x_{1} - x_{2}\right| $ juga bergantung kepada nilai diskriminan. Namun, jika diskriminannya negatif, $ \left|x_{1} - x_{2}\right| $ tidak terdefinisi.
	
	\par Pertukaran variabel antara $ x_{1} $ dan $ x_{2} $ juga tidak akan memengaruhi nilai dari $ x_{1} + x_{2} $ dan $ x_{1}x_{2} $ karena berlaku sifat komutatif. Inilah yang disebut sebagai kesimetrisan ekspresi aljabar. Artinya, pertukaran variabel $ x_{1} $ dan $ x_{2} $ tidak akan berpengaruh terhadap nilainya. Oleh karena itu, $ x_{1} + x_{2} $ dan $ x_{1}x_{2} $ disebut sebagai ekspresi aljabar yang simetris.
	
	\begin{explbox}
		Apakah $ \left|x_{1} - x_{2}\right| $ juga simetris? Bagaimana dengan $ x_{1} - x_{2} $?
	\end{explbox}

\subsection{Sifat-sifat Akar Persamaan Kuadrat}
	
	Studi mengenai diskriminan membawa kita ke dalam pemahaman yang lebih dalam mengenai sifat-sifat akar persamaan kuadrat. Tentunya terkadang kita ingin membatasi solusi suatu persamaan kuadrat yang diberikan. Entah itu kita menginginkan kedua akarnya positif, atau setidaknya satu akar positif, hal ini akan sangat dibutuhkan nantinya ketika mendalami matematika lebih lanjut, bahkan di bidang keilmuan lainnya. Fisikawan pasti akan kebingungan apabila persamaan kuadrat yang berkaitan mengenai massa memberikannya solusi negatif.
	
	\begin{catatan}
		Banyak sifat-sifat pertidaksamaan disini yang mungkin agak dibingungkan oleh pembaca. Untuk saat ini, kita 'meminjamnya' terlebih dahulu tanpa pembahasan. Oleh karena itu, Anda dapat membaca terlebih dahulu dasar-dasar mengenai pertidaksamaan pada awalan bab 3 untuk dapat membantu Anda memahami bagian ini.
	\end{catatan}
	
	\par Dari sini, kita misalkan $ x_{1} $ dan $ x_{2} $ sebagai akar-akar dari suatu persamaan kuadrat. Sifat-sifat akar persamaan kuadrat adalah sebagai berikut.
	
	\subsubsection{Kedua Akarnya Bernilai Positif}
	
		Karena kedua akarnya positif, maka $ x_{1} > 0 $ dan $ x_{2} > 0 $. Oleh karena itu, dengan menggunakan teorema \ref{thm:21}, haruslah
		\[ \frac{-b + \sqrt{b^{2} - 4ac}}{2a} > 0 \quad \mbox{dan} \quad \frac{-b - \sqrt{b^{2} - 4ac}}{2a} > 0 \]
		atau dengan mensubstitusi $ D = b^{2} - 4ac $, menyelesaikan pertidaksamaan terakhir sama saja dengan menyelesaikan pertidaksamaan
		\[ \frac{-b + \sqrt{D}}{2a} > 0 \quad \mbox{dan} \quad \frac{-b - \sqrt{D}}{2a} > 0. \]
		Meskipun telah mensubstitusi/memisalkan $ b^{2} - 4ac $ sebagai $ D $, pertidaksamaan yang harus diselesaikan tetap saja sangat sulit. Oleh karena itu, kita gunakan hasil yang telah kita dapatkan dari bagian sebelumnya mengenai bentuk-bentuk simetris akar-akar persamaan kuadrat.
		
		\par Perhatikan bahwa jika kita mengalikan pertidaksamaan $ x_{1} > 0 $ dengan $ x_{2} > 0 $, maka kita akan mendapatkan $ x_{1}x_{2} > 0 $ sehingga $ \dfrac{c}{a} > 0 $. Selain itu, jika kita menjumlahkan kedua pertidaksamaan tersebut, kita akan mendapatkan $ x_{1} + x_{2} > 0 $ sehingga $ -\dfrac{b}{a} > 0 $. Ingat bahwa agar kedua pertidaksamaan lebih besar daripada nol, maka kita perlu $ x_{1} $ dan $ x_{2} $ berupa bilangan real. Bilangan tidak real, seperti $ i = \sqrt{-1} $ tidak memenuhi sifat urutan\footnote{Sifat urutan ini akan dipelajari lebih lanjut pada mata kuliah Analisis Real. Meskipun demikian, pada buku ini juga terdapat pembahasannya secara umum pada bab 3.} (seperti lebih besar dari, kurang dari) sehingga kita tidak bisa mengatakan $ i > 0 $ atau $ i < 0 $. Oleh karena itu, agar keduanya merupakan bilangan real, haruslah $ D \geq 0 $ (mengapa bukan '$ D > 0 $'?).
		
		\par Jadi, syarat-syarat agar kedua akar suatu persamaan kuadrat bernilai positif adalah
		\begin{equation} \label{eq:216}
			x_{1} + x_{2} > 0, \quad x_{1}x_{2} > 0, \quad \mbox{dan} \quad D \geq 0
		\end{equation}
		atau
		\[ -\frac{b}{a} > 0, \quad \frac{c}{a} > 0, \quad \mbox{dan} \quad D \geq 0. \]
		
		\par Dari sini timbul permasalahan baru. Apakah sistem pertidaksamaan \ref{eq:216} benar-benar menjamin $ x_{1} > 0 $ dan $ x_{2} > 0 $? Kita telah memverifikasi bahwa haruslah $ D \geq 0 $. Kita hanya tinggal membuktikan bahwa $ x_{1} + x_{2} > 0 $ dan $ x_{1}x_{2} > 0 $ akan menjamin $ x_{1} > 0 $ dan $ x_{2} > 0 $.
		
		\par Perhatikan bahwa jika $ x_{1}x_{2} > 0 $, maka $ x_{1} $ dan $ x_{2} $ keduanya harus memiliki tanda yang sama, artinya keduanya harus sama-sama positif atau sama-sama negatif karena jika salah satunya negatif, maka $ x_{1}x_{2} $ akan bernilai negatif. Dari pertidaksamaan $ x_{1} + x_{2} > 0 $ dan fakta bahwa $ x_{1} $ dan $ x_{2} $ harus bertanda sama, maka sudah pasti $ x_{1} > 0 $ dan $ x_{2} > 0 $. Jika keduanya negatif tentu saja jumlahnya akan semakin negatif yang akan mengakibatkan kontradiksi.
	
	\subsubsection{Kedua Akarnya Bernilai Negatif}
		
		Karena kedua akarnya negatif, maka $ x_{1} < 0 $ dan $ x_{2} < 0 $. Oleh karena itu, haruslah $ x_{1} + x_{2} < 0 $ dan $ x_{1}x_{2} > 0 $. Selain itu, dari subbagian sebelumnya, agar keduanya negatif, haruslah $ x_{1} $ dan $ x_{2} $ bernilai real sehingga $ D \geq 0 $.
		
		\par Jadi, syarat-syarat agar kedua akar suatu persamaan bernilai negatif adalah
		\begin{equation} \label{eq:217}
			x_{1} + x_{2} < 0, \quad x_{1}x_{2} > 0, \quad \mbox{dan} \quad D \geq 0
		\end{equation}
		atau
		\[ -\frac{b}{a} < 0, \quad \frac{c}{a} > 0, \quad \mbox{dan} \quad D \geq 0. \]
	
	\subsubsection{Kedua Akarnya Berlainan/Berlawanan Tanda}
		
		Penggunaan kata berlawanan disini mungkin akan menyebabkan keambiguan karena pada subbagian selanjutnya terdapat istilah lain yang mengandung kata 'berlawanan' sehingga ada baiknya kita menggunakan kata 'berlainan' disini.
		
		\par Perhatikan bahwa karena kedua akarnya berlainan tanda, maka kemungkinannya adalah $ x_{1} < 0 $ dan $ x_{2} > 0 $, atau $ x_{1} > 0 $ dan $ x_{2} > 0 $. Dari kedua kemungkinan tersebut, pastilah berlaku $ x_{1}x_{2} > 0 $. Karena salah satu akar harus positif dan akar lainnya haruslah negatif, maka kedua akar haruslah merupakan bilangan real berbeda sehingga $ D > 0 $.
		
		\par Jadi, syarat-syarat agar kedua akar suatu persamaan kuadrat berlainan tanda adalah
		\begin{equation} \label{eq:218}
			x_{1}x_{2} < 0 \quad \mbox{dan} \quad D > 0
		\end{equation}
		atau
		\[ \frac{c}{a} < 0 \quad \mbox{dan} \quad D > 0. \]
		
		\begin{explbox}
			Mengapa tidak perlu ada syarat untuk $ x_{1} + x_{2} $?
		\end{explbox}
	
	\subsubsection{Kedua Akarnya Berlawanan}
		
		Subbagian inilah yang memiliki kata yang sama dengan subbagian sebelumnya. Disini, maksud dari kedua akarnya berlawanan adalah jika salah satu akarnya $ x_{1} $, maka akar lainnya adalah $ -x_{1} $ sehingga $ x_{2} = -x_{1} $.
		
		\par Karena $ x_{2} = -x_{1} $, maka $ x_{1} + x_{2} = 0 $. Disini, kita tidak perlu memberikan syarat untuk $ D $. Hal ini dikarenakan nilai dari $ x_{1} $ dan $ x_{2} $ tidak harus merupakan bilangan real, karena jika $ x_{1} = i $ dan $ x_{2} = -i $, maka $ x_{1} + x_{2} = i - i = 0 $ yang juga memenuhi kondisi kedua akarnya berlawanan. Selain itu, kita juga tidak memerlukan syarat untuk $ x_{1}x_{2} $. Hal ini dikarenakan syarat $ x_{1} + x_{2} = 0 $ sudah akan dapat menjamin $ x_{2} = -x_{1} $ (yaitu dengan mengurangi kedua ruas dengan $ x_{2} $).
		
		\par Jadi, syarat agar kedua akar suatu persamaan kuadrat saling berlawanan adalah
		\begin{equation} \label{eq:219}
			x_{1} + x_{2} < 0
		\end{equation}
		atau
		\[ -\frac{b}{a} < 0 \]
	
	\subsubsection{Kedua Akarnya Berkebalikan}
		
		Kedua akar suatu persamaan kuadrat dikatakan saling berkebalikan jika $ x_{2} = \dfrac{1}{x_{1}} $ atau sebaliknya. Oleh karena itu, haruslah $ x_{1}x_{2} = 1 $.
		
		\par Jadi, syarat agar kedua akar suatu persamaan kuadrat saling berkebalikan adalah
		\begin{equation} \label{eq:220}
			x_{1}x_{2} = 1
		\end{equation}
		atau
		\[ \frac{c}{a} = 1 \]
		
		\begin{explbox}
			Mengapa tidak perlu ada syarat untuk $ x_{1} + x_{2} $ dan $ D $?
		\end{explbox}
	
	\begin{contoh}
		Tentukan nilai dari $ m $ agar persamaan kuadrat $ x^{2} - 2x - m + 1 = 0 $ memiliki
		\begin{enumerate}
			\item akar-akar yang bernilai positif.
			\item akar-akar yang bernilai negatif.
			\item akar-akar yang berlainan tanda.
			\item akar-akar yang berlawanan tanda.
			\item akar-akar yang berkebalikan.
		\end{enumerate}
	\end{contoh}
	\begin{jawab}
		Perhatikan bahwa $ a = 1 $, $ b = -2 $, dan $ c = -m + 1 $. Pertama-tama, kita cari terlebih dahulu nilai dari $ x_{1} + x_{2} $, $ x_{1}x_{2} $, dan $ D $. Dengan menggunakan persamaan \ref{eq:210} dan \ref{eq:211}, kita punyai
		\[ x_{1} + x_{2} = -\frac{b}{a} = -\frac{-2}{1} = 2 \quad \mbox{dan} \quad x_{1}x_{2} = \frac{c}{a} = \frac{-m + 1}{1} = -m + 1. \]
		Selain itu, dengan menggunakan persamaan \ref{eq:208}, kita juga punyai
		\[ D = b^{2} - 4ac = \left(-2\right)^{2} - 4\left(1\right)\left(-m + 1\right) = 4 + 4m - 4 = 4m. \]
		Selanjutnya, kita akan menyelesaikan soal ini.
		\begin{enumerate}
			\item Dari sistem pertidaksamaan \ref{eq:216}, syarat-syarat agar kedua akar bernilai positif adalah $ x_{1} + x_{2} = 2 > 0 $ (sehingga bernilai benar untuk semua $ m \in \mathbb{R} $), $ x_{1}x_{2} = -m + 1 > 0 \iff m < 1 $, dan $ D = 4m \geq 0 \iff m \geq 0 $. Oleh karena itu, agar ketiga syarat terpenuhi, haruslah $ m $ memenuhi irisan dari ketiga interval tadi
			\begin{minipage}[t]{\linewidth}
				\begin{figure}[H]
					\centering
					\begin{tikzpicture}
						\begin{axis}[
							axis x line=middle,
							axis y line=none,
							axis line style=<->,
							xmin=-3,
							ymin=0,
							height=2.5cm,
							width=8cm
							]
							{
								\addplot[domain=-3:1, thick, blue, <-] {1};
								\addplot[color=blue, holdot] coordinates{(1, 1)};
								
								\addplot[domain=0:4, thick, red, ->] {2};
								\addplot[color=red, soldot] coordinates{(0, 2)};
							}
						\end{axis}
					\end{tikzpicture}
					\caption{Ilustrasi interval $ m < 1 $ (biru) dan $ m \geq 0 $ (merah) jika digambarkan pada garis bilangan real. Disini, syarat pertama, yaitu $ m \in \mathbb{R} $ tidak perlu digambar (mengapa?).}
				\end{figure}
			\end{minipage}
			sehingga $ \HP = \set{m \in \mathbb{R}}{0 \leq m < 1} $ atau jika dituliskan dalam notasi interval, $ m \in \lkrb{0, 1} $.
			\item Latihan, jawab dalam notasi himpunan dan notasi interval.
			\item Latihan, jawab dalam notasi himpunan dan notasi interval.
			\item Latihan.
			\item Latihan.
		\end{enumerate}
	\end{jawab}

	\begin{explbox}
		Diberikan bilangan real positif $ a $. Tentukan syarat-syarat agar kedua akar suatu persamaan kuadrat lebih besar dari $ a $. Buktikan jawaban Anda.
	\end{explbox}

\subsection{Akar Tunggal vs Akar Ganda}
	
	Ketika suatu persamaan kuadrat yang akar-akarnya $ x_{1} $ dan $ x_{2} $ memiliki diskriminan nol, maka $ x_{1} = x_{2} $. Telah menjadi perdebatan yang cukup lama mengenai jumlah akar persamaan kuadrat yang diskriminannya nol. Ada beberapa pendapat yang berkata bahwa akarnya tunggal, ada juga beberapa pendapat yang menyebutkan bahwa akarnya tetap dua, tetapi ganda.
	
	\par Kedua pendapat tersebut benar tergantung darimana kita melihatnya. Pertama, pendapat yang mengatakan bahwa akarnya tunggal menilai bahwa jika suatu persamaan kuadrat memiliki diskriminan nol, seperti pada persamaan $ x^{2} - 4x + 4 = 0 $, maka ruas kiri persamaan tersebut dapat difaktorkan menjadi $ \left(x - 2\right)^{2} = 0 $ sehingga $ x - 2 = 0 $ yang memiliki tepat satu solusi, yaitu $ x = 2 $. Kedua, pendapat yang mengatakan bahwa akarnya tetap dua, tetapi ganda, menilai bahwa ketika menarik akar kedua ruas, $ x $ harus diganti dengan $ x_{1, 2} $ seperti pada proses membuktikan rumus kuadratik, yaitu ketika $ \left(x - 2\right)^{2} = 0 $ ditarik akarnya pada kedua ruas, haruslah $ x_{1, 2} - 2 = 0 \iff x_{1, 2} = 2 $ sehingga $ x_{1} = x_{2} = 2 $ yang tetap memiliki tepat dua solusi, tetapi solusinya ganda.
	
	\par Untuk menyelesaikan hal ini agar tidak menjadi perdebatan yang berkepanjangan, kita memiliki istilah \textit{multiplisitas} dalam menyelesaikannya. Sebelumnya, tinjau pernyataan "bentuk kuadrat $ \left(x - a\right)\left(x - b\right) $ memiliki dua akar". Pernyataan tersebut benar jika $ a \ne b $ tetapi tidak untuk $ a = b $. Untuk kasus $ a = b $, maka bentuk kuadrat tersebut ekuivalen dengan $ \left(x - a\right)^{2} $ dan kita katakan $ a $ sebagai akar ganda. Dari sinilah diperkenalkan istilah multiplisitas, yaitu akar ganda dihitung dua kali, akar tripel dihitung tiga kali, dan seterusnya, sehingga sebarang polinomial berderajat $ n $ akan selalu memiliki $ n $ akar, termasuk polinomial berderajat dua, yaitu bentuk kuadrat.
	
	\par Oleh karena itu, persamaan kuadrat dengan diskriminan nol akan memiliki dua akar (yang tentunya ganda) jika kita menghitungnya dengan multiplisitasnya. Meskipun demikian dalam buku ini, kita sepakat untuk menggunakan istilah 'akar ganda' dibanding 'akar tunggal', meski tidak secara eksplisit menyebutkan multiplisitas dalam pernyataannya.
	
	\par Karenanya, jumlah akar (atau hasil kalinya) persamaan kuadrat yang memiliki diskriminan nol dapat kita hitung menggunakan formula \ref{eq:210} dengan aman. Sebagai contoh, jumlah akar persamaan kuadrat $ x^{2} - 4x + 4 = 0 $ adalah $ \dfrac{-4}{1} = -4 $, meskipun akarnya sebenarnya hanya $ x = 2 $ jika kita menghitungnya tanpa multiplisitas (yaitu jika kita menganggapnya sebagai akar tunggal).

\subsection{Menyusun Persamaan Kuadrat Baru}
	
	Sebelumnya, kita tahu bahwa persamaan kuadrat
	\[ \left(x - x_{1}\right)\left(x - x_{2}\right) = 0 \]
	merupakan persamaan kuadrat yang akar-akarnya $ x_{1} $ dan $ x_{2} $. Jika kita menjabarkan ruas kiri persamaan terakhir, kita akan mendapatkan
	\[ x^{2} - x_{2}x - x_{1}x + x_{1}x_{2} = 0 \]
	sehingga
	\begin{equation} \label{eq:221}
		x^{2} - \left(x_{1} + x_{2}\right)x + x_{1}x_{2} = 0.
	\end{equation}
	Sebagai catatan, proses penjabaran ini juga merupakan bukti lain untuk formula \ref{eq:210} dan formula \ref{eq:211}.
	
	\par Oleh karena itu, jika kita mengetahui bahwa suatu persamaan kuadrat memiliki akar-akar $ x_{1} = 2 $ dan $ x_{2} = 3 $, kita langsung saja memasukkan nilai $ x_{1} $ dan $ x_{2} $ ke dalam persamaan \ref{eq:221} sehingga persamaan kuadrat yang memenuhi adalah
	\[ x^{2} - \left(2 + 3\right)x + 2\left(3\right) = 0 \iff x^{2} - 5x + 6 = 0. \]
	Anda dapat mengecek bahwa persamaan ini memiliki solusi $ x = 2 $ atau $ x = 3 $.
	
	\par Dengan menggunakan fakta tersebut, kita akan dengan mudah membentuk suatu persamaan kuadrat baru berdasarkan akar-akar dari persamaan kuadrat sebelumnya, meskipun kita tidak mengetahui akar-akarnya secara eksplisit. Cukup digunakan bentuk-bentuk simetris akar-akar persamaan kuadrat. Sebagai contoh, kita akan mengerjakan soal berikut.
	
	\begin{contoh}
		Diberikan persamaan kuadrat $ 5x^{2} - x + 1 = 0 $ yang akar-akarnya $ p $ dan $ q $. Tentukan persamaan kuadrat baru yang akar-akarnya $ p + 1 $ dan $ q + 1 $.
	\end{contoh}
	\begin{jawab}
		Tentunya jika kita mencari akar-akarnya dengan menggunakan teorema \ref{thm:21}, kemudian mencari $ p + 1 $ dan $ q + 1 $ secara manual, akan sangat menghabiskan waktu. Belum lagi persamaan kuadrat tersebut memiliki diskriminan negatif (cek!). Oleh karena itu, kita gunakan metode yang telah dijabarkan sebelumnya.
		\par Misalkan $ x_{1} = p + 1 $ dan $ x_{2} = q + 1 $ merupakan akar-akar dari persamaan kuadrat baru. Maka dengan menggunakan persamaan \ref{eq:221}, persamaan kuadrat yang akar-akarnya $ x_{1} $ dan $ x_{2} $ adalah
		\begin{alignat*}{2}
			&\qquad& x^{2} - \left(x_{1} + x_{2}\right)x + x_{1}x_{2} &= 0 \\
			\iff&& x^{2} - \left(\left(p + 1\right) + \left(q + 1\right)\right)x + \left(p + 1\right)\left(q + 1\right) &= 0 \\
			\iff&& x^{2} - \left(p + q + 2\right)x + \left(pq + p + q + 1\right) &= 0
		\end{alignat*}
		Padahal $ p $ dan $ q $ memenuhi $ p + q = \dfrac{1}{5} $ dan $ pq = \dfrac{1}{5} $ sehingga persamaan terakhir dapat ditulis ulang menjadi
		\[ x^{2} - \left(\frac{1}{5} + 2\right)x + \left(\frac{1}{5} + \frac{1}{5} + 1\right) = 0 \iff x^{2} - \frac{11}{5}x + \frac{7}{5} = 0. \]
		Oleh karena itu, mengalikan kedua ruas dengan 5 akan didapatkan
		\[ 5x^{2} - 11x + 7 = 0. \]
		Jadi, persamaan kuadrat baru yang akar-akarnya $ p + 1 $ dan $ q + 1 $ adalah
		\[ 5x^{2} - 11x + 7 = 0. \]
	\end{jawab}
	
	\par Untuk lebih memantapkan lagi, kita akan mengerjakan satu contoh lain sebagai berikut.
	
	\begin{contoh}
		Diberikan persamaan kuadrat $ x^{2} - 5x + 10 = 0 $ yang akar-akarnya $ r $ dan $ s $. Tentukan persamaan kuadrat baru yang akar-akarnya $ r^{2} $ dan $ s^{2} $.
	\end{contoh}
	\begin{jawab}
		Persamaan kuadrat baru yang akar-akarnya $ r^{2} $ dan $ s^{2} $ adalah
		\[ x^{2} - \left(r^{2} + s^{2}\right)x + r^{2} \cdot s^{2} = 0 \iff x^{2} - \left(\left(r + s\right)^{2} - 2rs\right)x + \left(rs\right)^{2} = 0. \]
		Padahal $ r $ dan $ s $ memenuhi $ r + s = 5 $ dan $ rs = 10 $ sehingga persamaan terakhir dapat ditulis ulang menjadi
		\[ x^{2} - \left(5^{2} - 2\left(10\right)\right)x + 10^{2} = 0 \iff x^{2} - 5x + 100 = 0. \]
		Jadi, persamaan kuadrat baru yang akar-akarnya $ r^{2} $ dan $ s^{2} $ adalah
		\[ x^{2} - 5x + 100 = 0. \]
	\end{jawab}

\subsection{Contoh Soal HOTS}
	
	Subbab ini tentunya harus diakhiri dengan memberikan contoh soal-soal HOTS agar pembaca dapat lebih menguasai materi-materi yang diberikan pada subbab ini. Pembaca diharapkan dapat mencoba untuk mengerjakan contoh-contoh ini terlebih dahulu sebelum membaca solusinya.
	
	\begin{contoh}
		Tentukan banyaknya penyelesaian real dari persamaan
		\[ x^{4} - 5x^{3} + 6x^{2} - 5x + 1 = 0. \]
	\end{contoh}
	\begin{jawab}
		Setelah melihat soalnya, pasti kita cenderung berpikir bahwa ini bukanlah merupakan persamaan kuadrat. Selain itu, persamaan ini juga sangat sulit untuk diselesaikan karena belum dipelajari mengenai teknik menyelesaikan persamaan polinomial berderajat empat. Tetapi, dengan sedikit eksplorasi, kita akan mendapatkan solusi yang lebih mudah tanpa mengetahui teknik-teknik lanjutan. Hanya cukup beberapa pemfaktoran dan pemisalan, kita akan dapat menjawab soal ini dengan menggunakan pengetahuan persamaan kuadrat yang telah kita pelajari sebelumnya.
		\par \noindent Pertama-tama, kita kelompokkan terlebih tahulu suku-suku yang memiliki koefisien sama sehingga
		\[ x^{4} - 5x^{3} + 6x^{2} - 5x + 1 = 0 \iff \left(x^{4} + 1\right) - 5\left(x^{3} + x\right) + 6x^{2} = 0. \]
		Dengan memfaktorkan keluar $ x^{2} $, persamaan terakhir dapat kita tuliskan sebagai
		\[ x^{2}\left(\left(x^{2} + \frac{1}{x^{2}}\right) - 5\left(x + \frac{1}{x}\right) + 6\right) = 0. \]
		Perhatikan bahwa $ x = 0 $ bukanlah penyelesaian dari persamaan pada soal (cek!) sehingga kita dapat membagi kedua ruas dengan $ x^{2} $. Oleh karena itu kita punyai
		\[ \left(x^{2} + \frac{1}{x^{2}}\right) - 5\left(x + \frac{1}{x}\right) + 6 = 0. \]
		Selanjutnya, misalkan $ x + \dfrac{1}{x} = u $, maka
		\[ u^{2} = \left(x + \frac{1}{x}\right)^{2} = x^{2} + 2 + \frac{1}{x^{2}} \iff u^{2} - 2 = x^{2} + \frac{1}{x^{2}} \]
		sehingga
		\begin{alignat*}{2}
			&\qquad& \left(x^{2} + \frac{1}{x^{2}}\right) - 5\left(x + \frac{1}{x}\right) + 6 &= 0 \\
			\iff&& u^{2} - 2 - 5u + 6 &= 0 \\
			\iff&& u^{2} - 5u + 4 &= 0 \\
			\iff&& \left(u - 1\right)\left(u - 4\right) &= 0
		\end{alignat*}
		Oleh karena itu, terdapat dua kasus untuk ditinjau.
		\par \noindent \textit{Kasus 1.} $ u - 1 = 0 \iff u = 1 $. \\
		Dengan melakukan substitusi balik nilai $ u $ akan didapatkan
		\[ x + \frac{1}{x} = 1 \iff x^{2} - x + 1 = 0. \]
		Perhatikan bahwa diskriminan persamaan terakhir adalah $ D = \left(-1\right)^{2} - 4\left(1\right)\left(1\right) = -3 $ sehingga untuk kasus ini tidak tidak ada solusi real $ x $ yang memenuhi.
		\par \noindent \textit{Kasus 2.} $ u - 4 = 0 \iff u = 4 $. \\
		Dengan melakukan substitusi balik nilai $ u $ akan didapatkan
		\[ x + \frac{1}{x} = 4 \iff x^{2} - 4x + 1 = 0. \]
		Perhatikan bahwa diskriminan persamaan terakhir adalah $ D = \left(-4\right)^{2} - 4\left(1\right)\left(1\right) = 12 $ sehingga untuk kasus ini memiliki dua solusi real yang berbeda.
		\par \noindent Jadi, banyaknya penyelesaian real dari persamaan $ x^{4} - 5x^{3} + 6x^{2} - 5x + 1 = 0 $ adalah 2.
	\end{jawab}
	
	\begin{contoh}
		Persamaan kuadrat $ x^{2} + ax + b + 1 = 0 $ dengan $ a $, $ b $ adalah bilangan bulat, memiliki akar-akar bilangan asli. Buktikan bahwa $ a^{2} + b^{2} $ bukan merupakan bilangan prima.
	\end{contoh}
	\begin{jawab}
		Misalkan akar-akar persamaan kuadrat pada soal adalah $ x_{1} $ dan $ x_{2} $, maka dengan formula \ref{eq:210} dan formula \ref{eq:211}, akan didapatkan
		$$ x_{1} + x_{2} = -a \quad \mbox{dan} \quad x_{1}x_{2} = b + 1 $$
		atau
		$$ -\left(x_{1} + x_{2}\right) = a \quad \mbox{dan} \quad b = x_{1}x_{2} - 1 $$
		sehingga
		\begin{align*}
			a^{2} + b^{2} &= \left(-\left(x_{1} + x_{2}\right)\right)^{2} + \left(x_{1}x_{2} - 1\right)^{2} \\
			&= \left(x_{1} + x_{2}\right)^{2} + \left(x_{1}x_{2} - 1\right)^{2} \\
			&= x_{1}^{2} + 2x_{1}x_{2} + x_{2}^{2} + x_{1}^{2}x_{2}^{2} - 2x_{1}x_{2} + 1 \\
			&= x_{1}^{2} + x_{2}^{2} + x_{1}^{2}x_{2}^{2} + 1 \\
			&= \left(x_{1}^{2} + 1\right)\left(x_{2}^{2} + 1\right)
		\end{align*}
		Karena akar-akar persamaan kuadrat pada soal merupakan bilangan asli, maka $ x_{1}^{2} $ dan $ x_{2}^{2} $ juga pasti bilangan asli sehingga $ x_{1}^{2} + 1 $ dan $ x_{2}^{2} + 1 $ merupakan bilangan asli yang lebih dari satu.
		\par \noindent Karena $ a^{2} + b^{2} $ dapat dituliskan sebagai perkalian dua bilangan asli yang lebih besar dari satu, maka jelas bahwa $ a^{2} + b^{2} $ bukan merupakan bilangan prima.
		\par \noindent Jadi terbukti bahwa $ a^{2} + b^{2} $ bukan merupakan bilangan prima. \hfill $ \square $
	\end{jawab}

\newpage

\subsection{Latihan Soal 2.1}
	
	\subsubsection{Bagian Pertama}
		
		\begin{enumerate}[nosep]
			\item Tentukan mana saja yang merupakan persamaan kuadrat, berikan alasan Anda:
			\begin{enumerate}
				\item $ m^{2} = 0 $
				\item $ x^{4} + x^{2} - 2 = 0 $
				\item $ y^{5} = 3 $
				\item $ \delta^{2} - 3\delta + 10 = 0 $
				\item $ p^{2} - \sqrt{p} + 5 = 0 $
				\item $ x^{4} + ax^{3} + bx^{2} + ax + 1 = 0 $ untuk suatu konstanta positif $ a $ dan $ b $.
			\end{enumerate}
			\item Tentukan nilai dari $ a $, $ b $, dan $ c $ dari persamaan kuadrat dalam $ x $ berikut:
			\begin{enumerate}
				\item $ 5x^{2} - 10x - 3 = 0 $
				\item $ x^{2} = -1 $
				\item $ x^{2} + 2x = 0 $
				\item $ -5x^{2} - px - 10p^{2} + 10p - 1 = 0 $
				\item $ \left(m - 1\right)x^{2} - 10x + px - 20p + 1 = 0 $
			\end{enumerate}
			\item Tunjukkan bahwa $ x = 3 $ adalah akar dari persamaan $ x^{2} + 9x - 36 = 0 $ tetapi $ x = 10 $ bukan.
			\item Salah satu akar dari persamaan kuadrat $ ax^{2} - \left(a + 6\right)x - 35 = 0 $ adalah $ x = \dfrac{7}{2} $. Jika ada, tentukanlah nilai dari akar yang lain.
			\item Jika $ u $ dan $ v $ adalah akar-akar dari persamaan $ x^{2} - 5x + 13 = 0 $, tentukanlah nilai dari
			\begin{enumerate}
				\item $ \left(u^{2} - 5u\right)\left(v^{2} - 5v\right) $.
				\item $ \left(2u^{2} - 10v + 5\right) + \left(3v^{2} - 15v + 17\right) $.
			\end{enumerate}
			\item Tentukanlah akar-akar persamaan di bawah ini dengan cara memfaktorkan.
			\begin{multcols}
				\begin{enumerate}
					\item $ x^{2} - 3x + 2 = 0 $
					\item $ 4x^{2} - 5x = 0 $
					\item $ 7x^{2} = 1 $
					\item $ 10x^{2} - 9x - 7 = 0 $
					\item $ -7x^{2} - 6x + 1 = 0 $
					\item $ -6x^{2} - 7x + 3 = 0 $
				\end{enumerate}
			\end{multcols}
			\item Tentukanlah persamaan kuadrat yang akar-akarnya
			\begin{multcols}
				\begin{enumerate}
					\item 2 dan $ -\dfrac{1}{2} $
					\item 0 dan $ -1 $
					\item $ \dfrac{10}{3} $ dan $ -\dfrac{10}{3} $
					\item $ \dfrac{1}{5} $ dan $ -\dfrac{7}{3} $.
				\end{enumerate}
			\end{multcols}
			\item Tentukanlah nilai dari $ k $, $ m $, dan $ n $ pada setiap kesamaan di bawah ini.
			\begin{enumerate}
				\item $ x^{2} + x + 1 = k\left(x + m\right)^{2} + n $
				\item $ 2x^{2} - 10x + 3 = k\left(x + m\right)^{2} + n $
				\item $ 3x^{2} - 7x - 18 = k\left(x + m\right)^{2} + n $
				\item $ x^{2} - 6x + 9 = k\left(x + m\right)^{2} + n $
			\end{enumerate}
			\item Tentukanlah akar-akar persamaan di bawah ini dengan cara melengkapkan kuadrat.
			\begin{multcols}
				\begin{enumerate}
					\item $ x^{2} - 7x + 10 = 0 $
					\item $ 2x^{2} - 5x - 9 = 0 $
					\item $ 7x^{2} + x - 1 = 0 $
					\item $ x^{2} + 4x + 4 = 0 $
					\item $ x^{2} + x + 1 = 0 $
					\item[]
				\end{enumerate}
			\end{multcols}
			\item Tentukanlah akar-akar persamaan kuadrat pada soal di atas dengan menggunakan rumus kuadratik (rumus abc).
			\item Tentukanlah diskriminan dari persamaan kuadrat berikut.
			\begin{multcols}
				\begin{enumerate}
					\item $ -x^{2} - 10x + 1 = 0 $
					\item $ 4 - 7x - x^{2} = 3 $
					\item $ 2x^{2} + 2x - 1 = 0 $
					\item $ 2x^{2} - 3x + 17 = 0 $
				\end{enumerate}
			\end{multcols}
			\item Buatlah 3 contoh persamaan kuadrat yang akar-akarnya takreal.
			\item Buatlah 3 contoh persamaan kuadrat yang akar-akarnya real, tetapi irasional.
			\item Persamaan kuadrat $ x^{2} + px + q = 0 $ memiliki dua solusi bilangan bulat untuk suatu bilangan prima $ p $ dan $ q $. Tentukan semua nilai $ p $ dan $ q $ yang mungkin.
			\item Diskriminan dari persamaan kuadrat $ -2x^{2} + 3x - 11\alpha = 0 $ adalah $ -12 $. Tentukanlah nilai dari $ \alpha $.
			\item Tentukan syarat nilai $ m $ agar persamaan kuadrat $ 3x^{2} + 2x - m + 1 = 0 $ memiliki
			\begin{multcols}
				\begin{enumerate}
					\item akar-akar real
					\item akar-akar real yang berbeda
					\item akar-akar takreal
					\item[]
				\end{enumerate}
			\end{multcols}
			\item Persamaan kuadrat $ x^{2} - 2mx + m - 1 = 0 $ memiliki akar kembar. Tentukanlah nilai dari $ m $.
			\item \probtype{PF} Tentukanlah syarat-syarat agar akar-akar suatu persamaan kuadrat merupakan bilangan rasional.
			\item Tentukanlah sekurang-kurangnya dua nilai $ m $ untuk setiap persamaan di bawah ini agar memiliki akar-akar rasional, lalu ujilah kebenarannya
			\begin{multcols}
				\begin{enumerate}
					\item $ 2x^{2} - 3x + m - 1 = 0 $
					\item $ -mx^{2} - x + 1 = 0 $
					\item $ 3x^{2} - mx + 4 = 0 $
					\item $ 2mx^{2} - 3x - m - 1 = 0 $
				\end{enumerate}
			\end{multcols}
			\item Jika $ x_{1} $ dan $ x_{2} $ adalah akar-akar dari persamaan kuadrat $ x^{2} - 2x + 7 = 0 $, maka tentukanlah nilai dari
			\begin{multcols}
				\begin{enumerate}
					\item $ x_{1}^{2} + x_{2}^{2} $
					\item $ x_{1}x_{2}^{3} + x_{1}^{3}x_{2} $
					\item $ x_{1}^{3} + x_{2}^{3} $
					\item $ \left|x_{1}^{2} - x_{2}^{2}\right| $
					\item $ \dfrac{1}{x_{1}} + \dfrac{1}{x_{2}} $
					\item $ \left(x_{1} - 1\right)\left(x_{2} + 1\right) $
					\item $ x_{1}^{-3} - x_{2}^{-3} $
					\item $ \dfrac{x_{1}}{x_{2}} + \dfrac{x_{2}}{x_{1}} $
					\item $ x_{1}^{5} + x_{2}^{5} $
					\item $ \dfrac{1}{\sqrt{x_{1}}} + \dfrac{1}{\sqrt{x_{2}}} $
				\end{enumerate}
			\end{multcols}
			\item Jika $ r $ dan $ s $ merupakan akar-akar dari persamaan $ 2x^{2} - 1x + 2 = 0 $, tentukanlah nilai dari
			\[ \frac{r}{\left(r + 1\right)^{2}} + \frac{s}{\left(s^{2} + 1\right)^{2}}. \]
			\item Jika $ p $ dan $ q $ merupakan akar-akar dari persamaan $ x^{2} - 2x + 3 = 0 $, tentukanlah nilai dari
			\[ \left(p^{2} - 4p + 2\right)\left(q^{2} + 4\right). \]
			\item Selisih akar-akar persamaan $ 5x^{2} - x + a = 0 $ adalah $ -3 $. Tentukanlah nilai dari $ a $.
			\item Diketahui persamaan kuadrat $ 3x^{2} - mx + m - 1 = 0 $ mempunyai akar-akar $ \alpha $ dan $ \beta $. Jika $ \alpha^{3} + \beta^{3} = 23 $, tentukanlah nilai dari $ m $.
			\item Tentukan banyaknya penyelesaian yang memenuhi sistem persamaan
			\[
				\begin{cases}
					x^{2} - ax + 2021 = 0 \\
					x^{2} - 2021x + a = 0
				\end{cases}
			\]
			untuk $ x < 0 $.
			\item Diberikan persamaan kuadrat
			\begin{enumerate}[label=(\roman*)]
				\item $ \left(a^{2} - 3\right)x^{2} + ax - 3 = 0 $
				\item $ x^{2} - x + 2a = 0 $
				\item $ x^{2} - \left(2a + 3\right)x - a^{2} + a + 1 = 0 $
			\end{enumerate}
			Tentukanlah syarat untuk $ a $ dalam setiap persamaan tersebut agar
			\begin{multcols}
				\begin{enumerate}
					\item akar-akarnya positif
					\item akar-akarnya negatif
					\item akar-akarnya berlainan tanda
					\item akar-akarnya berkebalikan
					\item akar-akarnya berlawanan
					\item[]
				\end{enumerate}
			\end{multcols}
			\item Kedua akar persamaan $ x^{2} - 3x + a - 3 = 0 $ lebih besar dari 2. Tentukanlah syarat untuk $ a $.
			\item Tentukan persamaan kuadrat baru yang akar-akarnya dua kali lebih besar dari akar-akar persamaan kuadrat $ 2x^{2} - 3x + 1 = 0 $.
			\item Tentukan persamaan kuadrat yang mempunyai akar $ a $ dan $ b $ sehingga
			\[ \frac{1}{a} + \frac{1}{b} = \frac{7}{10}. \]
			\item Misalkan $ m $ dan $ n $ adalah akar-akar persamaan kuadrat $ 3x^{2} - 5x + 1 = 0 $. Tentukanlah persamaan kuadrat yang mempunyai akar-akar $ m^{-2} + 1 $ dan $ n^{-2} + 1 $.
			\item Misalkan $ a $ dan $ b $ adalah akar-akar persamaan kuadrat $ -2x^{2} + x - 7 = 0 $. Tentukanlah persamaan kuadrat yang akar-akarnya $ ab $ dan $ a^{2} + b^{2} $.
			\item Dari soal di atas, tentukanlah persamaan kuadrat yang akar-akarnya $ \sqrt{a} $ dan $ \sqrt{b} $.
			\item Jika $ \alpha  + 2\beta = 5 $ dan $ \alpha\beta = -2 $, maka tentukanlah persamaan kuadrat yang akar-akarnya $ \dfrac{\alpha}{\alpha + 1} $ dan $ \dfrac{2\beta}{2\beta + 1} $.
			\item Persamaan kuadrat $ 2x^{2} - px + 1 = 0 $ dengan $ p > 0 $ mempunyai akar-akar $ \alpha $ dan $ \beta $. Jika $ x^{2} - 5x + q = 0 $ mempunyai akar-akar $ \dfrac{1}{\alpha^{2}} $ dan $ \dfrac{1}{\beta^{2}} $, maka tentukanlah nilai dari $ q - p $.
			\item Jika $ m $ dan $ n $ adalah akar-akar dari persamaan kuadrat $ 2x^{2} + x - 2 = 0 $, maka tentukanlah persamaan kuadrat yang akar-akarnya $ m^{3} - n^{2} $ dan $ n^{3} - m^{2} $.
		\end{enumerate}
	
	\subsubsection{Bagian Kedua \dashh Soal Tantangan}
		
		\begin{enumerate}[nosep]
			\item Misalkan $ a $, $ b $ dan $ c $ adalah tiga bilangan \textit{berbeda}. Jika ketiga bilangan tersebut merupakan bilangan asli satu digit, tentukanlah jumlah terbesar akar-akar persamaan $ \left(x - a\right)\left(x- b\right) + \left(x - b\right)\left(x - c\right) = 0 $ yang mungkin.
			\item Misalkan $ \alpha $ dan $ \beta $ akar-akar dari persamaan
			\[ x^{2} - 2px + p^{2} - 2p - 1 = 0. \]
			Cari semua bilangan real $ p $ sedemikian sehingga
			\[ \frac{1}{2}\frac{\left(\alpha - \beta\right)^{2} - 2}{\left(\alpha + \beta\right)^{2} + 2} \]
			merupakan bilangan bulat.
			\item Diketahui $ p $ dan $ q $ merupakan bilangan prima. Jika persamaan $ x^{2} - px + q = 0 $ memiliki akar-akar bilangan bulat positif yang berbeda, tentukanlah nilai dari $ p $ dan $ q $.
			\item Tinjau persamaan $ x^{2} + px + q = 0 $. Berapa banyak persamaan demikian yang memiliki akar-akar real jika $ p $ dan $ q $ hanya boleh dipilih dari himpunan $ \lrbr{1, 2, 3, 4, 5, 6} $?
			\item Diberikan persamaan kuadrat $ ax^{2} - bx + c = 0 $ dengan $ a $, $ b $, dan $ c $ semuanya merupakan bilangan asli. Jika persamaan kuadrat tersebut memiliki dua akar berbeda yang berada pada interval $ \left(0, 1\right) $, carilah nilai paling minimum yang mungkin dari $ abc $.
			\item \probtype{PF} Misalkan $ p $ dan $ q $ bilangan real sedemikian sehingga persamaan kuadrat $ x^{2} + px + q = 0 $ memiliki dua akar berbeda $ x_{1} $ dan $ x_{2} $. Asumsikan $ \left|x_{1} - x_{2}\right| = 1 $ dan $ \left|p - q\right| = 1 $. Buktikan bahwa $ p, q, x_{1}, x_{2} $ semuanya merupakan bilangan bulat.
			\item Jika $ x_{1} $ dan $ x_{2} $ adalah akar-akar dari persamaan kuadrat $ x^{2} + x - 3 = 0 $, tentukanlah nilai dari $ 4x_{1}^{2} + 3x_{2}^{2} + 2x_{1} + x_{2} $.
			\item Jika akar-akar persamaan $ x^{2} - 45x - 8 = 0 $ adalah $ \alpha $ dan $ \beta $, maka tentukanlah nilai dari $ \sqrt[3]{\alpha} + \sqrt[3]{\beta} $.
			\item Misalkan $ a $, $ b $, $ c $, dan $ d $ bilangan real taknol. Jika $ a $ dan $ b $ adalah solusi dari $ x^{2} + cx + d = 0 $ serta $ c $ dan $ d $ solusi dari $ x^{2} + ax + b = 0 $, maka tentukanlah nilai dari $ a + b + c + d $.
			\item Jika persamaan kuadrat $ x^{2} + ax + b = 0 $ dan $ x^{2} + px + q = 0 $ memiliki satu akar yang sama, maka tentukanlah persamaan kuadrat yang akar-akarnya merupakan akar-akar yang lain dari kedua persamaan kuadrat sebelumnya.
			\item Cari semua pasangan bilangan real $ a $ dan $ b $ dengan $ b > 0 $ sedemikian sehingga solusi dari dua persamaan
			\[ x^{2} + ax + a = b \quad \mbox{dan} \quad x^{2} + ax + a = -b \]
			merupakan empat bilangan bulat berurutan.
			\item \probtype{PF} Diketahui rasio dari akar-akar persamaan kuadrat $ \ell x^{2} + nx + n = 0 $ adalah $ p : q $. Buktikan atau bantah bahwa
			\[ \sqrt{\frac{p}{q}} + \sqrt{\frac{q}{p}} - \sqrt{\frac{n}{\ell}} = 0 \]
			\item Jika setiap dua dari tiga persamaan kuadrat
			\begin{align*}
				 x^{2} - a^{2}x + a + 1 &= 0, \\
				 x^{2} - \left(a + 1\right)x + a &= 0, \mbox{ dan} \\
				 x^{2} - 3ax + x + a^{2} + 2 &= 0
			\end{align*}
			selalu memiliki tepat satu akar yang sama, maka tentukanlah semua bilangan real $ a $ yang munkin.
			\item Jika $ \alpha $ dan $ \beta $ akar-akar dari persamaan kuadrat $ x^{2} - 2x - 5 = 0 $, maka tentukanlah nilai dari $ \alpha^{4} - 28\alpha $.
			\item Jika $ p $ dan $ q $ akar-akar dari persamaan $ x^{2} - x + 1 = 0 $, tentukanlah nilai dari $ p^{2021} + q^{2021} $.
			\item Diketahui $ x_{1} $ dan $ x_{2} $ adalah dua bilangan bulat berbeda yang merupakan akar-akar dari persamaan kuadrat $ x^{2} + px + q + 1 = 0 $. Jika $ p $ dan $ p^{2} + q^{2} $ adalah bilangan-bilangan prima, tentukan nilai terbesar yang mungkin dari $ x_{1}^{2021} + x_{2}^{2021} $.
			\item \probtype{*} Untuk bilangan real $ x $, notasi $ \floor{x} $ menyatakan bilangan bulat terbesar yang tidak lebih besar dari $ x $; sedangkan $ \ceil{x} $ menyatakan bilangan bulat terkecil yang tidak lebih kecil dari $ x $. Tentukan semua bilangan real $ x $ yang memenuhi
			\[ \floor{x}^{2} - 3x + \ceil{x} = 0. \]
			\item \probtype{*} Danu dan Dini sedang bermain suatu permainan. Pada awalnya, Danu memilih tiga bilangan real taknol. Dini kemudian menyusun ketiga bilangan tadi sebagai koefisien persamaan kuadrat
			\[ \rule{1ex}{.4pt}x^{2} + \rule{1ex}{.4pt}x + \rule{1ex}{.4pt} = 0. \]
			Danu memenangkan permainan jika dan hanya jika persamaan yang dihasilkan memiliki dua solusi rasional berbeda.
			\item \probtype{PF} \probtype{**} Asumsikan $ a, b, c, A, B, C $ semuanya bilangan real dengan $ a \ne 0 $ dan $ A \ne 0 $ sedemikian sehingga
			\[ \left|ax^{2} + bx + c\right| \leq \left|Ax^{2} + Bx + C\right| \]
			untuk setiap bilangan real $ x $. Buktikan bahwa
			\[ \left|b^{2} - 4ac\right| \leq \left|B^{2} - 4AC\right|. \]
			\item \probtype{PF} \probtype{**} Misalkan $ a $ dan $ b $ bilangan bulat positif sedemikian sehingga $ ab + 1 $ membagi $ a^{2} + b^{2} $. Buktikan bahwa $ \dfrac{a^{2} + b^{2}}{ab + 1} $ merupakan kuadrat dari suatu bilangan bulat.
		\end{enumerate}

\newpage

%% Subbab 2 %%

\section{Fungsi Kuadrat}

Sebelum membahas mengenai fungsi kuadrat, kita harus mengetahui terlebih dahulu apa itu fungsi. Disini kita hanya memberikan hal-hal yang penting mengenai fungsi karena pendalaman lebih lanjut mengenai fungsi akan dipelajari pada mata kuliah Pengantar Dasar Matematika. Secara informal, fungsi adalah sesuatu yang memerlukan input dan menghasilkan output. Bisa juga kita anggap fungsi sebagai mesin. Jika kita memasukkan bahan baku ke dalam mesin (input), maka hasilnya adalah barang jadi (output).

\par Salah satu contoh fungsi dalam matematika adalah $ \func{f}{x} = 2x + 1 $ dan $ \func{g}{x} = \sqrt{x} $. Pada fungsi $ f $, jika kita masukkan $ x = 2 $ (sebagai input), maka hasilnya adalah $ \func{f}{2} = 2\left(2\right) + 1 = 5 $ (outputnya). Selain itu, pada fungsi $ g $, jika kita masukkan $ x = 4 $, maka $ \func{g}{4} = \sqrt{4} = 2 $. Disini bisa kita lihat bahwa $ f $ dapat menerima segala macam bilangan real, tetapi fungsi $ g $ tidak bisa menerima segala macam bilangan real, karena jika $ x < 0 $, fungsi tersebut tidak terdefinisi pada garis bilangan real.

\par Dalam hal ini, himpunan semua input yang mungkin bagi suatu fungsi $ f $ kita katakan sebagai domain dari fungsi $ f $ dan himpunan semua output yang mungkin bagi suatu fungsi $ f $ dikatakan sebagai jangkauan (\textit{range}) dari fungsi $ f $. Domain fungsi $ f $ biasa dinotasikan sebagai $ \func{\Dom}{f} $ atau $ D_{f} $, dan jangkauan fungsi $ f $ biasa dinotasikan sebagai $ \func{\Ran}{f} $ atau $ R_{f} $. Pada contoh sebelumnya, $ \func{f}{x} = 2x + 1 $ memiliki domain $ \func{\Dom}{f} = \mathbb{R} $ dan jangkauan $ \func{\Ran}{f} = \mathbb{R} $, serta $ \func{g}{x} = \sqrt{x} $ memiliki domain $ \func{\Dom}{g} = \mathbb{R}_{\geq 0} $ dan jangkauan $ \func{\Ran}{g} = \mathbb{R}_{\geq 0} $ (mengapa?).

\begin{explbox}
	Coba cari tahu mengenai kodomain dan \textit{image} dari suatu fungsi. Apakah perbedaan antara jangkauan, kodomain, dan \textit{image}?
\end{explbox}

\par Jika $ f $ memiliki domain $ A $ dan jangkauan $ B $, maka $ f $ merupakan fungsi dari himpunan $ A $ ke himpunan $ B $. Secara simbolik, kita bisa tuliskan sebagai $ f \colon A \to B $. Tentunya notasi ini akan lebih mempermudah penulisannya. Perlu diketahui juga bahwa fungsi $ \func{g}{x} = x + 1 $ sebenarnya ekuivalen dengan fungsi $ \func{r}{y} = y + 1 $. Variabel yang digunakan dalam suatu fungsi sebenarnya tidak terlalu penting, yang penting dalam suatu fungsi adalah aturan pemetaannya itu sendiri atau formulanya.

\par Setelah mengetahui sekilas mengenai fungsi secara umum, kita kemudian siap untuk mempelajari fungsi kuadrat. Fungsi kuadrat memiliki bentuk umum
\begin{equation} \label{eq:222}
	\func{f}{x} = ax^{2} + bx + c
\end{equation}
dengan $ a, b, c $ semuanya bilangan real dan $ a \ne 0 $.

\par Bentuk umum fungsi kuadrat di atas mengingatkan kita kepada sesuatu yang familiar, yaitu persamaan kuadrat. Tetapi dalam persamaan kuadrat, salah satu ruasnya adalah nol. Salah satu contoh fungsi kuadrat adalah $ \func{f}{x} = x^{2} - 3x + 1 $. Disini, nilai $ a = 1 $, $ b = -3 $, dan $ c = 1 $. Selain itu, $ \func{g}{x} = x^{2} $ juga merupakan fungsi kuadrat dengan nilai $ a = 1 $, $ b = 0 $, dan $ c = 0 $.

\subsection{Grafik Fungsi Kuadrat}
	
	\kant[1-2]

\subsection{Menentukan Fungsi Kuadrat}
	
	\kant[3-4]
	
\subsection{Domain dan Jangkauan Fungsi Kuadrat}
	
	\kant[5-6]

\subsection{Definit Positif dan Definit Negatif}
	
	\kant[7-8]
	
\subsection{Sifat Fungsi Kuadrat Terhadap Fungsi Lainnya}
	
	\kant[9-10]
	
\subsection{Titik Tetap Fungsi Kuadrat}
	
	\kant[11-12]
	
\subsection{Penerapan Fungsi Kuadrat dalam Kehidupan Sehari-hari}
	
	\kant[13-14]
	
\subsection{Fungsi Kuadrat Umum*}
	
	\kant[15-16]
	
\subsection{Latihan Soal 2.2}
	
	\kant[17-18]

\newpage

%% Subbab 3 %%

\section{Uji Kompetensi Bab \ref{sec:second}}

\kant[6-10]
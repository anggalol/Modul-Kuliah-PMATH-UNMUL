\chapter{Polinomial}
\label{sec:sixth}

% pendahuluan bahas definisi, suku-suku polinom, polinomial monik, kesamaan
\kant[1-5]

\section{Derajat Polinomial dan Jenis-jenisnya}
% derajat polinom, jenis-jenis polinom berdasarkan derajatnya, grafik
\kant[6-10]

\section{Pembagian Polinomial}
% pembagian Euclid polinomial, pembagian bersusun panjang, sintetik, teorema sisa
\kant[11-15]

\section{Persamaan Polinomial}
% pemfaktoran pp (kesamaan, pembagian susun panjang, sintetik, teorema faktor), polinomial tak tereduksi,
% teorema akar rasional, diskriminan polinomial, teorema fundamental aljabar, teorema abel-ruffini,
% solusi persamaan kubik, teorema vieta, algoritma pencarian akar, tipe khusus persamaan polinomial,
% aturan tanda Descartes, macam-macam irreducible criterion (perron)
\kant[16-20]

\section{Fungsi Rasional dan Dekomposisi Pecahan Parsial}
% definisi, akar-akar, dekomposisi pecahan parsial (jelaskan pake video bprp)
\kant[21-25]

\section{Uji Kompetensi Bab \ref{sec:sixth}}
\kant[26]

